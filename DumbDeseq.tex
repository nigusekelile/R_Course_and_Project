% Options for packages loaded elsewhere
\PassOptionsToPackage{unicode}{hyperref}
\PassOptionsToPackage{hyphens}{url}
%
\documentclass[
]{article}
\usepackage{amsmath,amssymb}
\usepackage{iftex}
\ifPDFTeX
  \usepackage[T1]{fontenc}
  \usepackage[utf8]{inputenc}
  \usepackage{textcomp} % provide euro and other symbols
\else % if luatex or xetex
  \usepackage{unicode-math} % this also loads fontspec
  \defaultfontfeatures{Scale=MatchLowercase}
  \defaultfontfeatures[\rmfamily]{Ligatures=TeX,Scale=1}
\fi
\usepackage{lmodern}
\ifPDFTeX\else
  % xetex/luatex font selection
\fi
% Use upquote if available, for straight quotes in verbatim environments
\IfFileExists{upquote.sty}{\usepackage{upquote}}{}
\IfFileExists{microtype.sty}{% use microtype if available
  \usepackage[]{microtype}
  \UseMicrotypeSet[protrusion]{basicmath} % disable protrusion for tt fonts
}{}
\makeatletter
\@ifundefined{KOMAClassName}{% if non-KOMA class
  \IfFileExists{parskip.sty}{%
    \usepackage{parskip}
  }{% else
    \setlength{\parindent}{0pt}
    \setlength{\parskip}{6pt plus 2pt minus 1pt}}
}{% if KOMA class
  \KOMAoptions{parskip=half}}
\makeatother
\usepackage{xcolor}
\usepackage[margin=1in]{geometry}
\usepackage{color}
\usepackage{fancyvrb}
\newcommand{\VerbBar}{|}
\newcommand{\VERB}{\Verb[commandchars=\\\{\}]}
\DefineVerbatimEnvironment{Highlighting}{Verbatim}{commandchars=\\\{\}}
% Add ',fontsize=\small' for more characters per line
\usepackage{framed}
\definecolor{shadecolor}{RGB}{248,248,248}
\newenvironment{Shaded}{\begin{snugshade}}{\end{snugshade}}
\newcommand{\AlertTok}[1]{\textcolor[rgb]{0.94,0.16,0.16}{#1}}
\newcommand{\AnnotationTok}[1]{\textcolor[rgb]{0.56,0.35,0.01}{\textbf{\textit{#1}}}}
\newcommand{\AttributeTok}[1]{\textcolor[rgb]{0.13,0.29,0.53}{#1}}
\newcommand{\BaseNTok}[1]{\textcolor[rgb]{0.00,0.00,0.81}{#1}}
\newcommand{\BuiltInTok}[1]{#1}
\newcommand{\CharTok}[1]{\textcolor[rgb]{0.31,0.60,0.02}{#1}}
\newcommand{\CommentTok}[1]{\textcolor[rgb]{0.56,0.35,0.01}{\textit{#1}}}
\newcommand{\CommentVarTok}[1]{\textcolor[rgb]{0.56,0.35,0.01}{\textbf{\textit{#1}}}}
\newcommand{\ConstantTok}[1]{\textcolor[rgb]{0.56,0.35,0.01}{#1}}
\newcommand{\ControlFlowTok}[1]{\textcolor[rgb]{0.13,0.29,0.53}{\textbf{#1}}}
\newcommand{\DataTypeTok}[1]{\textcolor[rgb]{0.13,0.29,0.53}{#1}}
\newcommand{\DecValTok}[1]{\textcolor[rgb]{0.00,0.00,0.81}{#1}}
\newcommand{\DocumentationTok}[1]{\textcolor[rgb]{0.56,0.35,0.01}{\textbf{\textit{#1}}}}
\newcommand{\ErrorTok}[1]{\textcolor[rgb]{0.64,0.00,0.00}{\textbf{#1}}}
\newcommand{\ExtensionTok}[1]{#1}
\newcommand{\FloatTok}[1]{\textcolor[rgb]{0.00,0.00,0.81}{#1}}
\newcommand{\FunctionTok}[1]{\textcolor[rgb]{0.13,0.29,0.53}{\textbf{#1}}}
\newcommand{\ImportTok}[1]{#1}
\newcommand{\InformationTok}[1]{\textcolor[rgb]{0.56,0.35,0.01}{\textbf{\textit{#1}}}}
\newcommand{\KeywordTok}[1]{\textcolor[rgb]{0.13,0.29,0.53}{\textbf{#1}}}
\newcommand{\NormalTok}[1]{#1}
\newcommand{\OperatorTok}[1]{\textcolor[rgb]{0.81,0.36,0.00}{\textbf{#1}}}
\newcommand{\OtherTok}[1]{\textcolor[rgb]{0.56,0.35,0.01}{#1}}
\newcommand{\PreprocessorTok}[1]{\textcolor[rgb]{0.56,0.35,0.01}{\textit{#1}}}
\newcommand{\RegionMarkerTok}[1]{#1}
\newcommand{\SpecialCharTok}[1]{\textcolor[rgb]{0.81,0.36,0.00}{\textbf{#1}}}
\newcommand{\SpecialStringTok}[1]{\textcolor[rgb]{0.31,0.60,0.02}{#1}}
\newcommand{\StringTok}[1]{\textcolor[rgb]{0.31,0.60,0.02}{#1}}
\newcommand{\VariableTok}[1]{\textcolor[rgb]{0.00,0.00,0.00}{#1}}
\newcommand{\VerbatimStringTok}[1]{\textcolor[rgb]{0.31,0.60,0.02}{#1}}
\newcommand{\WarningTok}[1]{\textcolor[rgb]{0.56,0.35,0.01}{\textbf{\textit{#1}}}}
\usepackage{graphicx}
\makeatletter
\def\maxwidth{\ifdim\Gin@nat@width>\linewidth\linewidth\else\Gin@nat@width\fi}
\def\maxheight{\ifdim\Gin@nat@height>\textheight\textheight\else\Gin@nat@height\fi}
\makeatother
% Scale images if necessary, so that they will not overflow the page
% margins by default, and it is still possible to overwrite the defaults
% using explicit options in \includegraphics[width, height, ...]{}
\setkeys{Gin}{width=\maxwidth,height=\maxheight,keepaspectratio}
% Set default figure placement to htbp
\makeatletter
\def\fps@figure{htbp}
\makeatother
\setlength{\emergencystretch}{3em} % prevent overfull lines
\providecommand{\tightlist}{%
  \setlength{\itemsep}{0pt}\setlength{\parskip}{0pt}}
\setcounter{secnumdepth}{-\maxdimen} % remove section numbering
\ifLuaTeX
  \usepackage{selnolig}  % disable illegal ligatures
\fi
\usepackage{bookmark}
\IfFileExists{xurl.sty}{\usepackage{xurl}}{} % add URL line breaks if available
\urlstyle{same}
\hypersetup{
  pdftitle={DumbDeseq: RNA Expression Software - R Project},
  hidelinks,
  pdfcreator={LaTeX via pandoc}}

\title{DumbDeseq: RNA Expression Software - R Project}
\author{}
\date{\vspace{-2.5em}}

\begin{document}
\maketitle

\begin{Shaded}
\begin{Highlighting}[]
\FunctionTok{getwd}\NormalTok{()}
\end{Highlighting}
\end{Shaded}

\begin{verbatim}
## [1] "D:/HackBio/R_Course_and_Project"
\end{verbatim}

\begin{Shaded}
\begin{Highlighting}[]
\FunctionTok{setwd}\NormalTok{(}\StringTok{"D:/HackBio/R\_Course\_and\_Project"}\NormalTok{)}
\FunctionTok{getwd}\NormalTok{()}
\end{Highlighting}
\end{Shaded}

\begin{verbatim}
## [1] "D:/HackBio/R_Course_and_Project"
\end{verbatim}

\begin{Shaded}
\begin{Highlighting}[]
\NormalTok{link\_to\_dataset }\OtherTok{=} \StringTok{"https://gist.githubusercontent.com/stephenturner/806e31fce55a8b7175af/raw/1a507c4c3f9f1baaa3a69187223ff3d3050628d4/results.txt"}

\NormalTok{DumbDeseq }\OtherTok{\textless{}{-}} \FunctionTok{read.table}\NormalTok{(}\AttributeTok{file =}\NormalTok{ link\_to\_dataset, }\AttributeTok{header =} \ConstantTok{TRUE}\NormalTok{)}
\end{Highlighting}
\end{Shaded}

\subsubsection{Generating the summary of
Datasets}\label{generating-the-summary-of-datasets}

\begin{Shaded}
\begin{Highlighting}[]
\FunctionTok{summary}\NormalTok{(DumbDeseq)}
\end{Highlighting}
\end{Shaded}

\begin{verbatim}
##      Gene           log2FoldChange          pvalue            padj          
##  Length:16406       Min.   :-2.129000   Min.   :0.0000   Min.   :0.0003053  
##  Class :character   1st Qu.:-0.178875   1st Qu.:0.2771   1st Qu.:0.9994000  
##  Mode  :character   Median :-0.007856   Median :0.5393   Median :0.9994000  
##                     Mean   :-0.035081   Mean   :0.5208   Mean   :0.9763603  
##                     3rd Qu.: 0.136675   3rd Qu.:0.7735   3rd Qu.:0.9994000  
##                     Max.   : 1.540000   Max.   :1.0000   Max.   :1.0000000
\end{verbatim}

\subsubsection{Write to a CSV file}\label{write-to-a-csv-file}

\begin{Shaded}
\begin{Highlighting}[]
\FunctionTok{write.csv}\NormalTok{(DumbDeseq, }\StringTok{"DumbDeseq.csv"}\NormalTok{, }\AttributeTok{row.names =} \ConstantTok{FALSE}\NormalTok{)}
\end{Highlighting}
\end{Shaded}

\begin{Shaded}
\begin{Highlighting}[]
\FunctionTok{head}\NormalTok{(DumbDeseq)}
\end{Highlighting}
\end{Shaded}

\begin{verbatim}
##      Gene log2FoldChange    pvalue      padj
## 1    DOK6         0.5100 1.861e-08 0.0003053
## 2    TBX5        -2.1290 5.655e-08 0.0004191
## 3 SLC32A1         0.9003 7.664e-08 0.0004191
## 4  IFITM1        -1.6870 3.735e-06 0.0068090
## 5   NUP93         0.3659 3.373e-06 0.0068090
## 6 EMILIN2         1.5340 2.976e-06 0.0068090
\end{verbatim}

\begin{Shaded}
\begin{Highlighting}[]
\NormalTok{data }\OtherTok{\textless{}{-}}\NormalTok{ DumbDeseq}
\end{Highlighting}
\end{Shaded}

\subsubsection{Volcano Plot}\label{volcano-plot}

\begin{Shaded}
\begin{Highlighting}[]
\CommentTok{\# Prepare data for the plot}
\NormalTok{neg\_log10\_pvalue }\OtherTok{\textless{}{-}} \SpecialCharTok{{-}}\FunctionTok{log10}\NormalTok{(data}\SpecialCharTok{$}\NormalTok{pvalue)}

\CommentTok{\# Create the base plot}
\FunctionTok{plot}\NormalTok{(data}\SpecialCharTok{$}\NormalTok{log2FoldChange, neg\_log10\_pvalue,}
     \AttributeTok{col =} \StringTok{"gray"}\NormalTok{, }\AttributeTok{pch =} \DecValTok{20}\NormalTok{, }\AttributeTok{cex =} \FloatTok{0.7}\NormalTok{,}
     \AttributeTok{xlab =} \StringTok{"Log2 Fold Change"}\NormalTok{, }\AttributeTok{ylab =} \StringTok{"{-}Log10(p{-}value)"}\NormalTok{,}
     \AttributeTok{main =} \StringTok{"Volcano Plot"}\NormalTok{)}

\CommentTok{\# Highlight upregulated genes (Log2FC \textgreater{} 1 and pvalue \textless{} 0.01)}
\FunctionTok{points}\NormalTok{(data}\SpecialCharTok{$}\NormalTok{log2FoldChange[data}\SpecialCharTok{$}\NormalTok{log2FoldChange }\SpecialCharTok{\textgreater{}} \DecValTok{1} \SpecialCharTok{\&}\NormalTok{ data}\SpecialCharTok{$}\NormalTok{pvalue }\SpecialCharTok{\textless{}} \FloatTok{0.01}\NormalTok{],}
\NormalTok{       neg\_log10\_pvalue[data}\SpecialCharTok{$}\NormalTok{log2FoldChange }\SpecialCharTok{\textgreater{}} \DecValTok{1} \SpecialCharTok{\&}\NormalTok{ data}\SpecialCharTok{$}\NormalTok{pvalue }\SpecialCharTok{\textless{}} \FloatTok{0.01}\NormalTok{],}
       \AttributeTok{col =} \StringTok{"red"}\NormalTok{, }\AttributeTok{pch =} \DecValTok{20}\NormalTok{, }\AttributeTok{cex =} \FloatTok{0.7}\NormalTok{)}

\CommentTok{\# Highlight downregulated genes (Log2FC \textless{} {-}1 and pvalue \textless{} 0.01)}
\FunctionTok{points}\NormalTok{(data}\SpecialCharTok{$}\NormalTok{log2FoldChange[data}\SpecialCharTok{$}\NormalTok{log2FoldChange }\SpecialCharTok{\textless{}} \SpecialCharTok{{-}}\DecValTok{1} \SpecialCharTok{\&}\NormalTok{ data}\SpecialCharTok{$}\NormalTok{pvalue }\SpecialCharTok{\textless{}} \FloatTok{0.01}\NormalTok{],}
\NormalTok{       neg\_log10\_pvalue[data}\SpecialCharTok{$}\NormalTok{log2FoldChange }\SpecialCharTok{\textless{}} \SpecialCharTok{{-}}\DecValTok{1} \SpecialCharTok{\&}\NormalTok{ data}\SpecialCharTok{$}\NormalTok{pvalue }\SpecialCharTok{\textless{}} \FloatTok{0.01}\NormalTok{],}
       \AttributeTok{col =} \StringTok{"blue"}\NormalTok{, }\AttributeTok{pch =} \DecValTok{20}\NormalTok{, }\AttributeTok{cex =} \FloatTok{0.7}\NormalTok{)}
\end{Highlighting}
\end{Shaded}

\includegraphics{DumbDeseq_files/figure-latex/unnamed-chunk-7-1.pdf}

\subsubsection{Top 5 Upregulated Genes}\label{top-5-upregulated-genes}

\begin{Shaded}
\begin{Highlighting}[]
\NormalTok{upregulated\_genes }\OtherTok{\textless{}{-}} \FunctionTok{subset}\NormalTok{(data, log2FoldChange }\SpecialCharTok{\textgreater{}} \DecValTok{1} \SpecialCharTok{\&}\NormalTok{ pvalue }\SpecialCharTok{\textless{}} \FloatTok{0.01}\NormalTok{)}
\FunctionTok{head}\NormalTok{(upregulated\_genes[}\FunctionTok{order}\NormalTok{(}\SpecialCharTok{{-}}\NormalTok{upregulated\_genes}\SpecialCharTok{$}\NormalTok{log2FoldChange), ], }\DecValTok{5}\NormalTok{)}
\end{Highlighting}
\end{Shaded}

\begin{verbatim}
##       Gene log2FoldChange    pvalue     padj
## 21   DTHD1          1.540 5.594e-05 0.043710
## 6  EMILIN2          1.534 2.976e-06 0.006809
## 30    PI16          1.495 1.297e-04 0.077940
## 35 C4orf45          1.288 2.472e-04 0.115900
## 75 FAM180B          1.249 1.146e-03 0.239900
\end{verbatim}

\subsubsection{The functions of Top Upregualated
Genes}\label{the-functions-of-top-upregualated-genes}

\begin{enumerate}
\def\labelenumi{\arabic{enumi}.}
\item
  DTHD1 - DTHD1 (Death Domain Containing 1) is a Protein Coding gene.
  Diseases associated with DTHD1 include Aniridia 1 and Leber Plus
  Disease. An important paralog of this gene is PSMD10.
\item
  EMILIN2 - EMILIN2 (Elastin Microfibril Interfacer 2) is a Protein
  Coding gene. Diseases associated with EMILIN2 include Lichen Nitidus
  and Epidermolytic Acanthoma. Among its related pathways are Elastic
  fibre formation and Extracellular matrix organization. Gene Ontology
  (GO) annotations related to this gene include extracellular matrix
  constituent conferring elasticity. An important paralog of this gene
  is EMILIN1.
\item
  PI16 - PI16 (Peptidase Inhibitor 16) is a Protein Coding gene.
  Diseases associated with PI16 include Corneal Dystrophy, Meesmann, 1
  and Epiphyseal Dysplasia, Multiple, 1. Gene Ontology (GO) annotations
  related to this gene include peptidase inhibitor activity.
\item
  C4orf45 - Aliases for SPMIP2 Gene. SPMIP2 (Sperm Microtubule Inner
  Protein 2) is a Protein Coding gene. Diseases associated with SPMIP2
  include Hyperekplexia.
\item
  FAM180B - FAM180B (Family With Sequence Similarity 180 Member B) is a
  Protein Coding gene. Diseases associated with FAM180B include
  Borderline Leprosy and Mosaic Variegated Aneuploidy Syndrome. An
  important paralog of this gene is FAM180A.
\end{enumerate}

\subsubsection{Top 5 Downregulated
Genes}\label{top-5-downregulated-genes}

\begin{Shaded}
\begin{Highlighting}[]
\NormalTok{downregulated\_genes }\OtherTok{\textless{}{-}} \FunctionTok{subset}\NormalTok{(data, log2FoldChange }\SpecialCharTok{\textless{}} \SpecialCharTok{{-}}\DecValTok{1} \SpecialCharTok{\&}\NormalTok{ pvalue }\SpecialCharTok{\textless{}} \FloatTok{0.01}\NormalTok{)}
\FunctionTok{head}\NormalTok{(downregulated\_genes[}\FunctionTok{order}\NormalTok{(downregulated\_genes}\SpecialCharTok{$}\NormalTok{log2FoldChange), ], }\DecValTok{5}\NormalTok{)}
\end{Highlighting}
\end{Shaded}

\begin{verbatim}
##       Gene log2FoldChange    pvalue      padj
## 2     TBX5         -2.129 5.655e-08 0.0004191
## 4   IFITM1         -1.687 3.735e-06 0.0068090
## 10     TNN         -1.658 8.973e-06 0.0147200
## 12 COL13A1         -1.647 1.394e-05 0.0159200
## 13  IFITM3         -1.610 1.202e-05 0.0159200
\end{verbatim}

\subsubsection{The functions of Top Downregualated
Genes}\label{the-functions-of-top-downregualated-genes}

\begin{enumerate}
\def\labelenumi{\arabic{enumi}.}
\tightlist
\item
  TBX5 - TBX5 (T-Box Transcription Factor 5) is a Protein Coding gene.
  Diseases associated with TBX5 include Holt-Oram Syndrome and Atrial
  Septal Defect 1. Among its related pathways are Gene expression
  (Transcription) and Cardiac conduction. Gene Ontology (GO) annotations
  related to this gene include DNA-binding transcription factor activity
  and transcription factor binding. An important paralog of this gene is
  TBX4.
\item
  IFITM1 - IFITM1 (Interferon Induced Transmembrane Protein 1) is a
  Protein Coding gene. Diseases associated with IFITM1 include Influenza
  and West Nile Virus. Among its related pathways are Cytokine Signaling
  in Immune system and Antiviral mechanism by IFN-stimulated genes. Gene
  Ontology (GO) annotations related to this gene include obsolete signal
  transducer activity, downstream of receptor. An important paralog of
  this gene is IFITM3.
\item
  TNN - TNN (Tenascin N) is a Protein Coding gene. Diseases associated
  with TNN include Seckel Syndrome and Lipodystrophy, Congenital
  Generalized, Type 1. Among its related pathways are PI3K-Akt signaling
  pathway and Extracellular matrix organization. Gene Ontology (GO)
  annotations related to this gene include identical protein binding and
  integrin binding. An important paralog of this gene is TNC.
\item
  COL13A1 - COL13A1 (Collagen Type XIII Alpha 1 Chain) is a Protein
  Coding gene. Diseases associated with COL13A1 include Myasthenic
  Syndrome, Congenital, 19 and Presynaptic Congenital Myasthenic
  Syndromes. Among its related pathways are Collagen chain trimerization
  and Integrin Pathway. Gene Ontology (GO) annotations related to this
  gene include heparin binding. An important paralog of this gene is
  COL7A1.
\item
  IFITM3 - IFITM3 (Interferon Induced Transmembrane Protein 3) is a
  Protein Coding gene. Diseases associated with IFITM3 include
  Influenza, Severe and Influenza. Among its related pathways are
  Cytokine Signaling in Immune system and Antiviral mechanism by
  IFN-stimulated genes. An important paralog of this gene is IFITM2.
\end{enumerate}

\end{document}
